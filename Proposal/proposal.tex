\documentclass{article}
\author{
  Torrent Gorjon, Xavier\\
  \texttt{Xavier.TorrentGorjon@os3.nl}
}
\title{Protecting against relay attacks}
\begin{document}


\begin{titlepage}
\center

\textsc{\LARGE University of Amsterdam}\\[1.5cm]

\textsc{\Large Research Project I proposal}\\[0.5cm]

\textsc{\Huge \textbf Protecting against\\relay attacks}\\[1.5cm]


\begin{minipage}{0.5 \textwidth}
\begin{center} \large
Xavier Torrent Gorj\'{o}n\\
\emph{Xavier.TorrentGorjon@os3.nl}\\[0.5cm]
\end{center}
\end{minipage}\\[3cm]
{\large \today} 


\end{titlepage}


\newpage

\tableofcontents
\section{Introduction}
As technology progresses, many traditional devices are improved with wireless features, aimed to improve user's comfort. Examples of these improvements can be found in access control (car keys, company IDs, public transportation cards) or credit card payment systems, among others.


Most of these systems are vulnerable to relay attacks, in which an attacker simply forwards messages between parties. These are specially dangerous in some cases, as they can provide unauthorized access even if the communication is fully encrypted and protected.
\section{Related work}
Related work here.
\section{Research Questions}
Research Questions here.
\section{Approach}
Approach and methodology here.
\section{Planning}
Planing here.
\section{Expected results}
Expected results here.
\section{Ethical considerations}
Ethical paragraph here. Bonus stage!
\section{References}
\end{document}
