\documentclass{article}
\author{
  Torrent Gorjon, Xavier\\
  \texttt{Xavier.TorrentGorjon@os3.nl}
}
\title{Protecting against relay attacks forging increased distance reports}

\usepackage{graphicx}

\usepackage[backend=bibtex]{biblatex}

\bibliography{references}


\begin{document}


\begin{titlepage}
\center
\textsc{}\\[1cm]
\textsc{\LARGE University of Amsterdam}\\[1.5cm]

\textsc{\Large Research Project I}\\[0.5cm]

\textsc{\Huge \textbf Protecting against relay\\[0cm] attacks forging increased\\[0.5cm] distance reports}\\[1.5cm]

\includegraphics[scale=1]{images/uva.png}\\[1cm]

\begin{minipage}{0.5 \textwidth}
\begin{center} \large
Xavier Torrent Gorj\'{o}n\\
\emph{Xavier.TorrentGorjon@os3.nl}\\[0.5cm]
\end{center}
\end{minipage}\\[2cm]
{\large \today} 


\end{titlepage}


\newpage


\renewcommand{\abstractname}{\Large Abstract}
\begin{abstract}
Lorem ipsum dolor sit amet, consectetur adipiscing elit. Sed sed diam metus. Quisque velit urna, dictum vel eros eu, congue luctus augue. Nulla sit amet metus nec ipsum pretium vestibulum ut quis sem. Nullam malesuada risus ut rhoncus consequat. Fusce in hendrerit nibh. Morbi a magna nunc. In vel justo tincidunt, porttitor tellus in, porta lacus. Nulla posuere enim arcu, eget aliquet mauris dictum ornare. Morbi iaculis nec elit vitae rutrum. Nam posuere, risus sed semper finibus, augue lorem blandit ex, non tempor nunc tortor vel arcu. Cras non tortor ipsum. Suspendisse vestibulum molestie nibh, lacinia efficitur nunc luctus non. Phasellus sed nibh at est suscipit pulvinar. Cras eleifend ante et volutpat suscipit.
\end{abstract}

\newpage

\section{Introduction}

Communications between machines face many challenges when the transmitted information needs to be protected. Most communications can prove to be valuable attack points for third parties that want to recover, modify, block or otherwise manipulate the original message sent for personal profit. Part of these attacks can be prevented by using end to end encryption and signature of the data. However relay attacks cannot be prevented just by using cryptographic algorithms.

Relay attacks consist on the mere reception and replay of information. Although at first this might seem harmless, many systems become vulnerable if that relaying of information is not noticed. One scenario that can be used as an example of the threat these attacks represent are access control systems, in which a device is used to prove that a user is within a certain distance from a validator through a challenge-response protocol. On unprotected implementations of these access control systems, an attacker can relay the challenge from the validator to a valid user who is not in range and relay its answer back to the validator, effectively bypassing distance validation. Practical attacks on this kind of systems have been demonstrated on various studies, as in \cite{francillon2011relay, francis2010practical, hancke2005practical, markantonakis2012practical}.

In this paper, we will first discuss the relay attacks used to forge fake location positions, focusing on the countermeasures against them. We will then focus on attacks forging increased distance reports, the feasibility of these attacks and propose solutions to them. In Section 2 we will discuss the available literature on this topic. We will present on Section 3 a more detailed explanation on the research questions this project aims to answer. Following on Section 4 we will explain the methodology used in this study. Sections 5 and 6 will discuss the actual results from our research questions, after which we will provide conclusions about the results gathered in Section 7.

\section{Related Work}

Relay attacks have been for long, and continue to be, an extensive field of research, as technologies and devices are shifting to a more mobile-focused paradigm. Many old procedures are being enhanced with wireless features, such as credit cards and car keys.

There are many papers available presenting solutions to distance bounding problems, as \cite{brands1994distance, tu2007rfid, rasmussen2010realization}. All of these studies are used as a base for others in a constant iteration to improve the protocols. Practical studies in this field tend to test the vulnerabilities on real applications, such as \cite{francillon2011relay, francis2010practical, hancke2005practical, markantonakis2012practical, vandenbreekel2014relay}. Although all that research refers to forging decreased distance reports, it has been deeply useful to our research as an starting point and inspiration.

Later on, we will require to make some assumptions and justifications on our investigation based on the characteristics of GPS signals. Many studies focus on the feasibility of intentional attacks against GPS systems, as \cite{warner2003gps, wen2005countermeasures, jafarnia2012gps}. These studies conclude that, even though spoofing is hard with the solutions they propose, it is not impossible.

This study is closely related to the field of MANETs (Mobile Ad-hoc NETworks), and as such, literature available in this topic is of our interest. In particular, wormhole attacks (\cite{hu2006wormhole, maheshwari2007detecting}) are a specific type of relay attack that, while being different than the ones we will study in this document, provide insight to our investigation as they are closely related.


\section{Research Questions}

Talk about the deviation from the initial research questions [2 paragraphs max]

Present both research questions:

\emph{Feasibility of forged increased distance report attacks}\\

Explain the question and why is it important. [1-2 paragraphs]

\emph{Fighting forged increased distance reports}\\

Explain the question and an introduction to how it will be done. [1-2 paragraphs]

\section{Methodology}

Explain how this is a theoretical study and that great part of it corresponds to literature review and searching sources from which to investigate solutions. [1-3 paragraphs]

\section{Feasibility of forged increased distance report attacks}

Answer the first research question.

Introduction to distance bounding. Explain available methods and why readiofrequency is the most reliable method. [at least 1 paragraph for each one. 2 or 3 for radiofrequency]

- Signal intensity

- Ultrasound

- Radiofrequency (extended explanation on this one as it is the focus of our research)

Explain why faking decreased distance reports is not feasible but faking increased reports is (on our study case). [1 paragraph]

Explain why increasing the reported distance between two parties can be a problem. [1-3 paragraphs]

Explain the theoretical attack case. Introduce and notice its difficulties as a practical attack. [1-2 paragraphs]

Talk about other systems that are used for location (GPS, RADAR) and justify how we use them. [1 or 2 paragraphs for each one]

Explain the assumptions made [1 paragraph]
   Feasibility of the jamming and relaying
   ...?
   
Present diverse attack scenarios. [at least 1 paragraph per case]
   Drones (multiple cases)
     Cooperative working
     Area surveillance
   Automatically driven cars
   Boats and harbours
  ...?
For each case, state clearly the assumptions made and its limitations


\section{Fighting forged increased distance reports}

Present the solutions

Multiple antenas and shared knowledge [Various paragraphs (5+), this is the first solution]

\section{Conclusions}

Consider the theoretical assumptions of the project [1 paragraph]

Given the proposed solutions, introduce and explain the implications [2 or more paragraphs]

\section{Future Work}

Briefly talk about the need of a more practical study with real hardware [1 paragraph]

...?

\section{Acknowledgements}

[2 paragraphs max]

\printbibliography

** bibliography goes here **

** appendix goes here **

%%% APPENDIX

\end{document}
